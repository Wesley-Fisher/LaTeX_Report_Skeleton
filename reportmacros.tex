% Figure Label, Figure filename (...png), Caption, scale
\newcommand{\addfigure}[4]
{

\begin{figure}[H]
	\centering
	\setlength{\fboxsep}{0pt}%
	\setlength{\fboxrule}{1pt}%
	\fbox{%
\includegraphics[width=#4\linewidth]{./Figures/#2}%
         }
	\caption{#3}
	\label{fig:#1}
\end{figure}
}

% Table Label, Table filename, Caption
\newcommand{\addtable}[3]
{
	\begin{table}[H]
		\centering
		\caption{#3}
		\input{./Data/#2}

		\label{tab:#1}
	\end{table}
	}
	
% Table Label, Table filename, Caption
\newcommand{\addcsvtable}[3]
{
	\csvautolongtable[table head={
		\caption{#3} \label{tab:#1}\\
		\hline
		\csvlinetotablerow\\ \hline
		\endfirsthead
		\hline
		\endfoot}, respect all]{./Data/#2}
}
	
\newcommand{\addthumbnail}[2]
{
	\includegraphics[width=#2\linewidth]{./Figures/#1}%
}
	
% Command for referencing a figure
\newcommand{\reffig}[1]
{Figure \ref{fig:#1}}

\newcommand{\reftable}[1]
{Table \ref{tab:#1}}

\newcommand{\refsection}[1]
{Section \ref{#1}}

\newcommand{\refappendix}[1]
{Appendix \ref{#1}}

\definecolor{light-gray}{gray}{0.95}
%% language, filename, firstline, lastline, caption
\newcommand{\addcodesnippet}[5]
{
	\lstinputlisting[language=#1, firstline=#3, lastline=#4, caption=#5, breaklines=true,backgroundcolor = \color{light-gray}]{./Code/#2}
}

\newcommand{\addcodefile}[3]
{
	\lstinputlisting[language=#1, caption=#3, breaklines=true,backgroundcolor = \color{light-gray}]{./Code/#2}
}

